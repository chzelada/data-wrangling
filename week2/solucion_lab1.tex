\documentclass[]{article}
\usepackage{lmodern}
\usepackage{amssymb,amsmath}
\usepackage{ifxetex,ifluatex}
\usepackage{fixltx2e} % provides \textsubscript
\ifnum 0\ifxetex 1\fi\ifluatex 1\fi=0 % if pdftex
  \usepackage[T1]{fontenc}
  \usepackage[utf8]{inputenc}
\else % if luatex or xelatex
  \ifxetex
    \usepackage{mathspec}
  \else
    \usepackage{fontspec}
  \fi
  \defaultfontfeatures{Ligatures=TeX,Scale=MatchLowercase}
\fi
% use upquote if available, for straight quotes in verbatim environments
\IfFileExists{upquote.sty}{\usepackage{upquote}}{}
% use microtype if available
\IfFileExists{microtype.sty}{%
\usepackage{microtype}
\UseMicrotypeSet[protrusion]{basicmath} % disable protrusion for tt fonts
}{}
\usepackage[margin=1in]{geometry}
\usepackage{hyperref}
\hypersetup{unicode=true,
            pdftitle={Solucion Lab \#1},
            pdfborder={0 0 0},
            breaklinks=true}
\urlstyle{same}  % don't use monospace font for urls
\usepackage{color}
\usepackage{fancyvrb}
\newcommand{\VerbBar}{|}
\newcommand{\VERB}{\Verb[commandchars=\\\{\}]}
\DefineVerbatimEnvironment{Highlighting}{Verbatim}{commandchars=\\\{\}}
% Add ',fontsize=\small' for more characters per line
\usepackage{framed}
\definecolor{shadecolor}{RGB}{248,248,248}
\newenvironment{Shaded}{\begin{snugshade}}{\end{snugshade}}
\newcommand{\KeywordTok}[1]{\textcolor[rgb]{0.13,0.29,0.53}{\textbf{#1}}}
\newcommand{\DataTypeTok}[1]{\textcolor[rgb]{0.13,0.29,0.53}{#1}}
\newcommand{\DecValTok}[1]{\textcolor[rgb]{0.00,0.00,0.81}{#1}}
\newcommand{\BaseNTok}[1]{\textcolor[rgb]{0.00,0.00,0.81}{#1}}
\newcommand{\FloatTok}[1]{\textcolor[rgb]{0.00,0.00,0.81}{#1}}
\newcommand{\ConstantTok}[1]{\textcolor[rgb]{0.00,0.00,0.00}{#1}}
\newcommand{\CharTok}[1]{\textcolor[rgb]{0.31,0.60,0.02}{#1}}
\newcommand{\SpecialCharTok}[1]{\textcolor[rgb]{0.00,0.00,0.00}{#1}}
\newcommand{\StringTok}[1]{\textcolor[rgb]{0.31,0.60,0.02}{#1}}
\newcommand{\VerbatimStringTok}[1]{\textcolor[rgb]{0.31,0.60,0.02}{#1}}
\newcommand{\SpecialStringTok}[1]{\textcolor[rgb]{0.31,0.60,0.02}{#1}}
\newcommand{\ImportTok}[1]{#1}
\newcommand{\CommentTok}[1]{\textcolor[rgb]{0.56,0.35,0.01}{\textit{#1}}}
\newcommand{\DocumentationTok}[1]{\textcolor[rgb]{0.56,0.35,0.01}{\textbf{\textit{#1}}}}
\newcommand{\AnnotationTok}[1]{\textcolor[rgb]{0.56,0.35,0.01}{\textbf{\textit{#1}}}}
\newcommand{\CommentVarTok}[1]{\textcolor[rgb]{0.56,0.35,0.01}{\textbf{\textit{#1}}}}
\newcommand{\OtherTok}[1]{\textcolor[rgb]{0.56,0.35,0.01}{#1}}
\newcommand{\FunctionTok}[1]{\textcolor[rgb]{0.00,0.00,0.00}{#1}}
\newcommand{\VariableTok}[1]{\textcolor[rgb]{0.00,0.00,0.00}{#1}}
\newcommand{\ControlFlowTok}[1]{\textcolor[rgb]{0.13,0.29,0.53}{\textbf{#1}}}
\newcommand{\OperatorTok}[1]{\textcolor[rgb]{0.81,0.36,0.00}{\textbf{#1}}}
\newcommand{\BuiltInTok}[1]{#1}
\newcommand{\ExtensionTok}[1]{#1}
\newcommand{\PreprocessorTok}[1]{\textcolor[rgb]{0.56,0.35,0.01}{\textit{#1}}}
\newcommand{\AttributeTok}[1]{\textcolor[rgb]{0.77,0.63,0.00}{#1}}
\newcommand{\RegionMarkerTok}[1]{#1}
\newcommand{\InformationTok}[1]{\textcolor[rgb]{0.56,0.35,0.01}{\textbf{\textit{#1}}}}
\newcommand{\WarningTok}[1]{\textcolor[rgb]{0.56,0.35,0.01}{\textbf{\textit{#1}}}}
\newcommand{\AlertTok}[1]{\textcolor[rgb]{0.94,0.16,0.16}{#1}}
\newcommand{\ErrorTok}[1]{\textcolor[rgb]{0.64,0.00,0.00}{\textbf{#1}}}
\newcommand{\NormalTok}[1]{#1}
\usepackage{graphicx,grffile}
\makeatletter
\def\maxwidth{\ifdim\Gin@nat@width>\linewidth\linewidth\else\Gin@nat@width\fi}
\def\maxheight{\ifdim\Gin@nat@height>\textheight\textheight\else\Gin@nat@height\fi}
\makeatother
% Scale images if necessary, so that they will not overflow the page
% margins by default, and it is still possible to overwrite the defaults
% using explicit options in \includegraphics[width, height, ...]{}
\setkeys{Gin}{width=\maxwidth,height=\maxheight,keepaspectratio}
\IfFileExists{parskip.sty}{%
\usepackage{parskip}
}{% else
\setlength{\parindent}{0pt}
\setlength{\parskip}{6pt plus 2pt minus 1pt}
}
\setlength{\emergencystretch}{3em}  % prevent overfull lines
\providecommand{\tightlist}{%
  \setlength{\itemsep}{0pt}\setlength{\parskip}{0pt}}
\setcounter{secnumdepth}{0}
% Redefines (sub)paragraphs to behave more like sections
\ifx\paragraph\undefined\else
\let\oldparagraph\paragraph
\renewcommand{\paragraph}[1]{\oldparagraph{#1}\mbox{}}
\fi
\ifx\subparagraph\undefined\else
\let\oldsubparagraph\subparagraph
\renewcommand{\subparagraph}[1]{\oldsubparagraph{#1}\mbox{}}
\fi

%%% Use protect on footnotes to avoid problems with footnotes in titles
\let\rmarkdownfootnote\footnote%
\def\footnote{\protect\rmarkdownfootnote}

%%% Change title format to be more compact
\usepackage{titling}

% Create subtitle command for use in maketitle
\newcommand{\subtitle}[1]{
  \posttitle{
    \begin{center}\large#1\end{center}
    }
}

\setlength{\droptitle}{-2em}

  \title{Solucion Lab \#1}
    \pretitle{\vspace{\droptitle}\centering\huge}
  \posttitle{\par}
    \author{}
    \preauthor{}\postauthor{}
    \date{}
    \predate{}\postdate{}
  

\begin{document}
\maketitle

\subsection{Objetivos}\label{objetivos}

\begin{enumerate}
\def\labelenumi{\arabic{enumi}.}
\tightlist
\item
  Solucion del laboratorio de la union de tablas en un mismo directorio
\end{enumerate}

\subsection{Archivos de Texto}\label{archivos-de-texto}

\subsubsection{Cargando librerias}\label{cargando-librerias}

\begin{Shaded}
\begin{Highlighting}[]
\KeywordTok{require}\NormalTok{(dplyr)}
\end{Highlighting}
\end{Shaded}

\begin{verbatim}
## Loading required package: dplyr
\end{verbatim}

\begin{verbatim}
## 
## Attaching package: 'dplyr'
\end{verbatim}

\begin{verbatim}
## The following objects are masked from 'package:stats':
## 
##     filter, lag
\end{verbatim}

\begin{verbatim}
## The following objects are masked from 'package:base':
## 
##     intersect, setdiff, setequal, union
\end{verbatim}

\begin{Shaded}
\begin{Highlighting}[]
\KeywordTok{require}\NormalTok{(readxl)}
\end{Highlighting}
\end{Shaded}

\begin{verbatim}
## Loading required package: readxl
\end{verbatim}

\subsubsection{\texorpdfstring{Funcion
\texttt{list.files()}}{Funcion list.files()}}\label{funcion-list.files}

\paragraph{Asignando la lista de archivos de la carpeta a una
variable}\label{asignando-la-lista-de-archivos-de-la-carpeta-a-una-variable}

\begin{Shaded}
\begin{Highlighting}[]
\NormalTok{files <-}\StringTok{ }\KeywordTok{list.files}\NormalTok{(}\DataTypeTok{path =} \StringTok{"C:/Users/JR29/Documents/GitHub/data-wrangling/data/01"}\NormalTok{)}
\NormalTok{files}
\end{Highlighting}
\end{Shaded}

\begin{verbatim}
##  [1] "01-2017.xlsx" "02-2017.xlsx" "03-2017.xlsx" "04-2017.xlsx"
##  [5] "05-2017.xlsx" "06-2017.xlsx" "07-2017.xlsx" "08-2017.xlsx"
##  [9] "09-2017.xlsx" "10-2017.xlsx" "11-2017.xlsx"
\end{verbatim}

\subsubsection{\texorpdfstring{Funcion
\texttt{read\_excel()}}{Funcion read\_excel()}}\label{funcion-read_excel}

\paragraph{\texorpdfstring{Lectura del primer archivo de Excel de la
carpeta
\texttt{01}}{Lectura del primer archivo de Excel de la carpeta 01}}\label{lectura-del-primer-archivo-de-excel-de-la-carpeta-01}

\begin{Shaded}
\begin{Highlighting}[]
\NormalTok{inv <-}\StringTok{ }\KeywordTok{read_excel}\NormalTok{(files[}\DecValTok{1}\NormalTok{], }\CommentTok{#sheet = "sheet1"}
                  \DataTypeTok{col_types =} \StringTok{"text"}\NormalTok{)}
\KeywordTok{str}\NormalTok{(inv)}
\end{Highlighting}
\end{Shaded}

\begin{verbatim}
## Classes 'tbl_df', 'tbl' and 'data.frame':    192 obs. of  8 variables:
##  $ COD_VIAJE: chr  "10000001" "10000002" "10000003" "10000004" ...
##  $ CLIENTE  : chr  "EL PINCHE OBELISCO / Despacho a cliente" "TAQUERIA EL CHINITO |||Faltante" "TIENDA LA BENDICION / Despacho a cliente" "TAQUERIA EL CHINITO" ...
##  $ UBICACION: chr  "76002" "76002" "76002" "76002" ...
##  $ CANTIDAD : chr  "1200" "1433" "1857" "339" ...
##  $ PILOTO   : chr  "Fernando Mariano Berrio" "Hector Aragones Frutos" "Pedro Alvarez Parejo" "Angel Valdez Alegria" ...
##  $ Q        : chr  "300" "358.25" "464.25" "84.75" ...
##  $ CREDITO  : chr  "30" "90" "60" "30" ...
##  $ UNIDAD   : chr  "Camion Grande" "Camion Grande" "Camion Grande" "Panel" ...
\end{verbatim}

\subsubsection{Determinar el tipo de variable por
columna}\label{determinar-el-tipo-de-variable-por-columna}

\begin{Shaded}
\begin{Highlighting}[]
\CommentTok{#inv$COD_VIAJE se queda en texto}
\CommentTok{#inv$CLIENTE se queda en texto}
\NormalTok{inv}\OperatorTok{$}\NormalTok{UBICACION <-}\StringTok{ }\KeywordTok{as.factor}\NormalTok{(inv}\OperatorTok{$}\NormalTok{UBICACION)}
\NormalTok{inv}\OperatorTok{$}\NormalTok{CANTIDAD <-}\StringTok{ }\KeywordTok{as.numeric}\NormalTok{(inv}\OperatorTok{$}\NormalTok{CANTIDAD)}
\CommentTok{#inv$PILOTO se queda en texto}
\NormalTok{inv}\OperatorTok{$}\NormalTok{Q <-}\StringTok{ }\KeywordTok{as.numeric}\NormalTok{(inv}\OperatorTok{$}\NormalTok{Q)}
\NormalTok{inv}\OperatorTok{$}\NormalTok{CREDITO <-}\StringTok{ }\KeywordTok{as.factor}\NormalTok{(inv}\OperatorTok{$}\NormalTok{CREDITO)}
\NormalTok{inv}\OperatorTok{$}\NormalTok{UNIDAD <-}\StringTok{ }\KeywordTok{as.factor}\NormalTok{(inv}\OperatorTok{$}\NormalTok{UNIDAD)}
\KeywordTok{str}\NormalTok{(inv)}
\end{Highlighting}
\end{Shaded}

\begin{verbatim}
## Classes 'tbl_df', 'tbl' and 'data.frame':    192 obs. of  8 variables:
##  $ COD_VIAJE: chr  "10000001" "10000002" "10000003" "10000004" ...
##  $ CLIENTE  : chr  "EL PINCHE OBELISCO / Despacho a cliente" "TAQUERIA EL CHINITO |||Faltante" "TIENDA LA BENDICION / Despacho a cliente" "TAQUERIA EL CHINITO" ...
##  $ UBICACION: Factor w/ 2 levels "76001","76002": 2 2 2 2 1 1 2 1 2 2 ...
##  $ CANTIDAD : num  1200 1433 1857 339 1644 ...
##  $ PILOTO   : chr  "Fernando Mariano Berrio" "Hector Aragones Frutos" "Pedro Alvarez Parejo" "Angel Valdez Alegria" ...
##  $ Q        : num  300 358.2 464.2 84.8 411 ...
##  $ CREDITO  : Factor w/ 3 levels "30","60","90": 1 3 2 1 1 1 3 2 1 3 ...
##  $ UNIDAD   : Factor w/ 3 levels "Camion Grande",..: 1 1 1 3 1 1 1 1 1 1 ...
\end{verbatim}

\subsubsection{Seleccionando y nombrando columnas
importantes}\label{seleccionando-y-nombrando-columnas-importantes}

\begin{Shaded}
\begin{Highlighting}[]
\NormalTok{inv <-}\StringTok{ }\NormalTok{inv[,}\KeywordTok{c}\NormalTok{(}\StringTok{"COD_VIAJE"}\NormalTok{, }\StringTok{"CLIENTE"}\NormalTok{, }\StringTok{"UBICACION"}\NormalTok{, }\StringTok{"CANTIDAD"}\NormalTok{, }\StringTok{"PILOTO"}\NormalTok{, }\StringTok{"Q"}\NormalTok{, }\StringTok{"CREDITO"}\NormalTok{, }\StringTok{"UNIDAD"}\NormalTok{)]}
\CommentTok{#names(inv) <- c(vector con nombres de columnas)}
\end{Highlighting}
\end{Shaded}

\subsubsection{\texorpdfstring{Funcion
\texttt{substr()}}{Funcion substr()}}\label{funcion-substr}

\paragraph{\texorpdfstring{Agregando columna \texttt{MES} y
\texttt{ANIO}}{Agregando columna MES y ANIO}}\label{agregando-columna-mes-y-anio}

\begin{Shaded}
\begin{Highlighting}[]
\NormalTok{inv}\OperatorTok{$}\NormalTok{MES <-}\StringTok{ }\KeywordTok{as.factor}\NormalTok{(}\KeywordTok{substr}\NormalTok{(files[}\DecValTok{1}\NormalTok{],}\DecValTok{1}\NormalTok{,}\DecValTok{2}\NormalTok{))}
\NormalTok{inv}\OperatorTok{$}\NormalTok{ANIO <-}\StringTok{ }\KeywordTok{as.factor}\NormalTok{(}\KeywordTok{substr}\NormalTok{(files[}\DecValTok{1}\NormalTok{],}\DecValTok{4}\NormalTok{,}\DecValTok{7}\NormalTok{))}
\KeywordTok{str}\NormalTok{(inv)}
\end{Highlighting}
\end{Shaded}

\begin{verbatim}
## Classes 'tbl_df', 'tbl' and 'data.frame':    192 obs. of  10 variables:
##  $ COD_VIAJE: chr  "10000001" "10000002" "10000003" "10000004" ...
##  $ CLIENTE  : chr  "EL PINCHE OBELISCO / Despacho a cliente" "TAQUERIA EL CHINITO |||Faltante" "TIENDA LA BENDICION / Despacho a cliente" "TAQUERIA EL CHINITO" ...
##  $ UBICACION: Factor w/ 2 levels "76001","76002": 2 2 2 2 1 1 2 1 2 2 ...
##  $ CANTIDAD : num  1200 1433 1857 339 1644 ...
##  $ PILOTO   : chr  "Fernando Mariano Berrio" "Hector Aragones Frutos" "Pedro Alvarez Parejo" "Angel Valdez Alegria" ...
##  $ Q        : num  300 358.2 464.2 84.8 411 ...
##  $ CREDITO  : Factor w/ 3 levels "30","60","90": 1 3 2 1 1 1 3 2 1 3 ...
##  $ UNIDAD   : Factor w/ 3 levels "Camion Grande",..: 1 1 1 3 1 1 1 1 1 1 ...
##  $ MES      : Factor w/ 1 level "01": 1 1 1 1 1 1 1 1 1 1 ...
##  $ ANIO     : Factor w/ 1 level "2017": 1 1 1 1 1 1 1 1 1 1 ...
\end{verbatim}

\subsubsection{Ciclo para union de todos los
archivos}\label{ciclo-para-union-de-todos-los-archivos}

\begin{Shaded}
\begin{Highlighting}[]
\NormalTok{i=}\DecValTok{2}
\ControlFlowTok{while}\NormalTok{(i}\OperatorTok{<=}\KeywordTok{length}\NormalTok{(files))\{}
\NormalTok{  g <-}\StringTok{ }\KeywordTok{read_excel}\NormalTok{(files[i]) }\CommentTok{#sheet = "sheet1"}
\NormalTok{  g <-}\StringTok{ }\NormalTok{g[,}\KeywordTok{c}\NormalTok{(}\StringTok{"COD_VIAJE"}\NormalTok{, }\StringTok{"CLIENTE"}\NormalTok{, }\StringTok{"UBICACION"}\NormalTok{, }\StringTok{"CANTIDAD"}\NormalTok{, }\StringTok{"PILOTO"}\NormalTok{, }\StringTok{"Q"}\NormalTok{, }\StringTok{"CREDITO"}\NormalTok{, }\StringTok{"UNIDAD"}\NormalTok{)]}
\NormalTok{  g}\OperatorTok{$}\NormalTok{MES <-}\StringTok{ }\KeywordTok{substr}\NormalTok{(files[i],}\DecValTok{1}\NormalTok{,}\DecValTok{2}\NormalTok{)}
\NormalTok{  g}\OperatorTok{$}\NormalTok{ANIO <-}\StringTok{ }\KeywordTok{substr}\NormalTok{(files[i],}\DecValTok{4}\NormalTok{,}\DecValTok{7}\NormalTok{)}
  \CommentTok{#names(g) <- c(vector con nombres de columnas)}
\NormalTok{  inv <-}\StringTok{ }\KeywordTok{rbind}\NormalTok{(inv,g)}
\NormalTok{  i=i}\OperatorTok{+}\DecValTok{1}
\NormalTok{\}}
\KeywordTok{str}\NormalTok{(inv)}
\end{Highlighting}
\end{Shaded}

\begin{verbatim}
## Classes 'tbl_df', 'tbl' and 'data.frame':    2180 obs. of  10 variables:
##  $ COD_VIAJE: chr  "10000001" "10000002" "10000003" "10000004" ...
##  $ CLIENTE  : chr  "EL PINCHE OBELISCO / Despacho a cliente" "TAQUERIA EL CHINITO |||Faltante" "TIENDA LA BENDICION / Despacho a cliente" "TAQUERIA EL CHINITO" ...
##  $ UBICACION: Factor w/ 2 levels "76001","76002": 2 2 2 2 1 1 2 1 2 2 ...
##  $ CANTIDAD : num  1200 1433 1857 339 1644 ...
##  $ PILOTO   : chr  "Fernando Mariano Berrio" "Hector Aragones Frutos" "Pedro Alvarez Parejo" "Angel Valdez Alegria" ...
##  $ Q        : num  300 358.2 464.2 84.8 411 ...
##  $ CREDITO  : Factor w/ 3 levels "30","60","90": 1 3 2 1 1 1 3 2 1 3 ...
##  $ UNIDAD   : Factor w/ 3 levels "Camion Grande",..: 1 1 1 3 1 1 1 1 1 1 ...
##  $ MES      : Factor w/ 11 levels "01","02","03",..: 1 1 1 1 1 1 1 1 1 1 ...
##  $ ANIO     : Factor w/ 1 level "2017": 1 1 1 1 1 1 1 1 1 1 ...
\end{verbatim}

\begin{Shaded}
\begin{Highlighting}[]
\KeywordTok{head}\NormalTok{(inv)}
\end{Highlighting}
\end{Shaded}

\begin{verbatim}
## # A tibble: 6 x 10
##   COD_VIAJE CLIENTE   UBICACION CANTIDAD PILOTO     Q CREDITO UNIDAD MES  
##   <chr>     <chr>     <fct>        <dbl> <chr>  <dbl> <fct>   <fct>  <fct>
## 1 10000001  EL PINCH~ 76002        1200. Ferna~ 300.  30      Camio~ 01   
## 2 10000002  TAQUERIA~ 76002        1433. Hecto~ 358.  90      Camio~ 01   
## 3 10000003  TIENDA L~ 76002        1857. Pedro~ 464.  60      Camio~ 01   
## 4 10000004  TAQUERIA~ 76002         339. Angel~  84.8 30      Panel  01   
## 5 10000005  CHICHARR~ 76001        1644. Juan ~ 411.  30      Camio~ 01   
## 6 10000006  UBIQUO L~ 76001        1827. Luis ~ 457.  30      Camio~ 01   
## # ... with 1 more variable: ANIO <fct>
\end{verbatim}

\subsubsection{Resumen de costo por Mes}\label{resumen-de-costo-por-mes}

\begin{Shaded}
\begin{Highlighting}[]
\NormalTok{inv }\OperatorTok\StringTok{ }\NormalTok{dplyr}\OperatorTok{::}\KeywordTok{group_by}\NormalTok{(MES) }\OperatorTok\StringTok{ }\NormalTok{dplyr}\OperatorTok{::}\KeywordTok{summarise}\NormalTok{(}\StringTok{`}\DataTypeTok{Costo Total}\StringTok{`}\NormalTok{ =}\StringTok{ }\KeywordTok{sum}\NormalTok{(Q))}
\end{Highlighting}
\end{Shaded}

\begin{verbatim}
## # A tibble: 11 x 2
##    MES   `Costo Total`
##    <fct>         <dbl>
##  1 01           55416.
##  2 02           56226.
##  3 03           48467.
##  4 04           51708.
##  5 05           60075.
##  6 06           52262.
##  7 07           56683.
##  8 08           53262.
##  9 09           51519.
## 10 10           55180.
## 11 11           58050.
\end{verbatim}

\subsubsection{Resumen de costo por
Ubicacion}\label{resumen-de-costo-por-ubicacion}

\begin{Shaded}
\begin{Highlighting}[]
\NormalTok{inv }\OperatorTok\StringTok{ }\NormalTok{dplyr}\OperatorTok{::}\KeywordTok{group_by}\NormalTok{(UBICACION) }\OperatorTok\StringTok{ }\NormalTok{dplyr}\OperatorTok{::}\KeywordTok{summarise}\NormalTok{(}\StringTok{`}\DataTypeTok{Costo Total}\StringTok{`}\NormalTok{ =}\StringTok{ }\KeywordTok{sum}\NormalTok{(Q))}
\end{Highlighting}
\end{Shaded}

\begin{verbatim}
## # A tibble: 2 x 2
##   UBICACION `Costo Total`
##   <fct>             <dbl>
## 1 76001           298277.
## 2 76002           300571.
\end{verbatim}

\subsubsection{Resumen de costo por
Unidad}\label{resumen-de-costo-por-unidad}

\begin{Shaded}
\begin{Highlighting}[]
\NormalTok{inv }\OperatorTok\StringTok{ }\NormalTok{dplyr}\OperatorTok{::}\KeywordTok{group_by}\NormalTok{(UNIDAD) }\OperatorTok\StringTok{ }\NormalTok{dplyr}\OperatorTok{::}\KeywordTok{summarise}\NormalTok{(}\StringTok{`}\DataTypeTok{Costo Total}\StringTok{`}\NormalTok{ =}\StringTok{ }\KeywordTok{sum}\NormalTok{(Q))}
\end{Highlighting}
\end{Shaded}

\begin{verbatim}
## # A tibble: 3 x 2
##   UNIDAD         `Costo Total`
##   <fct>                  <dbl>
## 1 Camion Grande        455466.
## 2 Camion Pequeño       112815.
## 3 Panel                 30566.
\end{verbatim}

\subsubsection{Resumen de costo por Mes, Ubicacion y
Unidad}\label{resumen-de-costo-por-mes-ubicacion-y-unidad}

\begin{Shaded}
\begin{Highlighting}[]
\NormalTok{inv }\OperatorTok\StringTok{ }\NormalTok{dplyr}\OperatorTok{::}\KeywordTok{group_by}\NormalTok{(MES, UBICACION, UNIDAD) }\OperatorTok\StringTok{ }\NormalTok{dplyr}\OperatorTok{::}\KeywordTok{summarise}\NormalTok{(}\StringTok{`}\DataTypeTok{Costo Total}\StringTok{`}\NormalTok{ =}\StringTok{ }\KeywordTok{sum}\NormalTok{(Q))}
\end{Highlighting}
\end{Shaded}

\begin{verbatim}
## # A tibble: 66 x 4
## # Groups:   MES, UBICACION [?]
##    MES   UBICACION UNIDAD         `Costo Total`
##    <fct> <fct>     <fct>                  <dbl>
##  1 01    76001     Camion Grande         21941.
##  2 01    76001     Camion Pequeño         5112.
##  3 01    76001     Panel                  1263.
##  4 01    76002     Camion Grande         23266.
##  5 01    76002     Camion Pequeño         2863.
##  6 01    76002     Panel                   972.
##  7 02    76001     Camion Grande         21638.
##  8 02    76001     Camion Pequeño         4531.
##  9 02    76001     Panel                  1308.
## 10 02    76002     Camion Grande         21584.
## # ... with 56 more rows
\end{verbatim}


\end{document}
