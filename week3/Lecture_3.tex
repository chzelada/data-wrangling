\documentclass[]{article}
\usepackage{lmodern}
\usepackage{amssymb,amsmath}
\usepackage{ifxetex,ifluatex}
\usepackage{fixltx2e} % provides \textsubscript
\ifnum 0\ifxetex 1\fi\ifluatex 1\fi=0 % if pdftex
  \usepackage[T1]{fontenc}
  \usepackage[utf8]{inputenc}
\else % if luatex or xelatex
  \ifxetex
    \usepackage{mathspec}
  \else
    \usepackage{fontspec}
  \fi
  \defaultfontfeatures{Ligatures=TeX,Scale=MatchLowercase}
\fi
% use upquote if available, for straight quotes in verbatim environments
\IfFileExists{upquote.sty}{\usepackage{upquote}}{}
% use microtype if available
\IfFileExists{microtype.sty}{%
\usepackage{microtype}
\UseMicrotypeSet[protrusion]{basicmath} % disable protrusion for tt fonts
}{}
\usepackage[margin=1in]{geometry}
\usepackage{hyperref}
\hypersetup{unicode=true,
            pdftitle={Semana 2},
            pdfauthor={Juan Carlos Girón},
            pdfborder={0 0 0},
            breaklinks=true}
\urlstyle{same}  % don't use monospace font for urls
\usepackage{color}
\usepackage{fancyvrb}
\newcommand{\VerbBar}{|}
\newcommand{\VERB}{\Verb[commandchars=\\\{\}]}
\DefineVerbatimEnvironment{Highlighting}{Verbatim}{commandchars=\\\{\}}
% Add ',fontsize=\small' for more characters per line
\usepackage{framed}
\definecolor{shadecolor}{RGB}{248,248,248}
\newenvironment{Shaded}{\begin{snugshade}}{\end{snugshade}}
\newcommand{\KeywordTok}[1]{\textcolor[rgb]{0.13,0.29,0.53}{\textbf{#1}}}
\newcommand{\DataTypeTok}[1]{\textcolor[rgb]{0.13,0.29,0.53}{#1}}
\newcommand{\DecValTok}[1]{\textcolor[rgb]{0.00,0.00,0.81}{#1}}
\newcommand{\BaseNTok}[1]{\textcolor[rgb]{0.00,0.00,0.81}{#1}}
\newcommand{\FloatTok}[1]{\textcolor[rgb]{0.00,0.00,0.81}{#1}}
\newcommand{\ConstantTok}[1]{\textcolor[rgb]{0.00,0.00,0.00}{#1}}
\newcommand{\CharTok}[1]{\textcolor[rgb]{0.31,0.60,0.02}{#1}}
\newcommand{\SpecialCharTok}[1]{\textcolor[rgb]{0.00,0.00,0.00}{#1}}
\newcommand{\StringTok}[1]{\textcolor[rgb]{0.31,0.60,0.02}{#1}}
\newcommand{\VerbatimStringTok}[1]{\textcolor[rgb]{0.31,0.60,0.02}{#1}}
\newcommand{\SpecialStringTok}[1]{\textcolor[rgb]{0.31,0.60,0.02}{#1}}
\newcommand{\ImportTok}[1]{#1}
\newcommand{\CommentTok}[1]{\textcolor[rgb]{0.56,0.35,0.01}{\textit{#1}}}
\newcommand{\DocumentationTok}[1]{\textcolor[rgb]{0.56,0.35,0.01}{\textbf{\textit{#1}}}}
\newcommand{\AnnotationTok}[1]{\textcolor[rgb]{0.56,0.35,0.01}{\textbf{\textit{#1}}}}
\newcommand{\CommentVarTok}[1]{\textcolor[rgb]{0.56,0.35,0.01}{\textbf{\textit{#1}}}}
\newcommand{\OtherTok}[1]{\textcolor[rgb]{0.56,0.35,0.01}{#1}}
\newcommand{\FunctionTok}[1]{\textcolor[rgb]{0.00,0.00,0.00}{#1}}
\newcommand{\VariableTok}[1]{\textcolor[rgb]{0.00,0.00,0.00}{#1}}
\newcommand{\ControlFlowTok}[1]{\textcolor[rgb]{0.13,0.29,0.53}{\textbf{#1}}}
\newcommand{\OperatorTok}[1]{\textcolor[rgb]{0.81,0.36,0.00}{\textbf{#1}}}
\newcommand{\BuiltInTok}[1]{#1}
\newcommand{\ExtensionTok}[1]{#1}
\newcommand{\PreprocessorTok}[1]{\textcolor[rgb]{0.56,0.35,0.01}{\textit{#1}}}
\newcommand{\AttributeTok}[1]{\textcolor[rgb]{0.77,0.63,0.00}{#1}}
\newcommand{\RegionMarkerTok}[1]{#1}
\newcommand{\InformationTok}[1]{\textcolor[rgb]{0.56,0.35,0.01}{\textbf{\textit{#1}}}}
\newcommand{\WarningTok}[1]{\textcolor[rgb]{0.56,0.35,0.01}{\textbf{\textit{#1}}}}
\newcommand{\AlertTok}[1]{\textcolor[rgb]{0.94,0.16,0.16}{#1}}
\newcommand{\ErrorTok}[1]{\textcolor[rgb]{0.64,0.00,0.00}{\textbf{#1}}}
\newcommand{\NormalTok}[1]{#1}
\usepackage{graphicx,grffile}
\makeatletter
\def\maxwidth{\ifdim\Gin@nat@width>\linewidth\linewidth\else\Gin@nat@width\fi}
\def\maxheight{\ifdim\Gin@nat@height>\textheight\textheight\else\Gin@nat@height\fi}
\makeatother
% Scale images if necessary, so that they will not overflow the page
% margins by default, and it is still possible to overwrite the defaults
% using explicit options in \includegraphics[width, height, ...]{}
\setkeys{Gin}{width=\maxwidth,height=\maxheight,keepaspectratio}
\IfFileExists{parskip.sty}{%
\usepackage{parskip}
}{% else
\setlength{\parindent}{0pt}
\setlength{\parskip}{6pt plus 2pt minus 1pt}
}
\setlength{\emergencystretch}{3em}  % prevent overfull lines
\providecommand{\tightlist}{%
  \setlength{\itemsep}{0pt}\setlength{\parskip}{0pt}}
\setcounter{secnumdepth}{0}
% Redefines (sub)paragraphs to behave more like sections
\ifx\paragraph\undefined\else
\let\oldparagraph\paragraph
\renewcommand{\paragraph}[1]{\oldparagraph{#1}\mbox{}}
\fi
\ifx\subparagraph\undefined\else
\let\oldsubparagraph\subparagraph
\renewcommand{\subparagraph}[1]{\oldsubparagraph{#1}\mbox{}}
\fi

%%% Use protect on footnotes to avoid problems with footnotes in titles
\let\rmarkdownfootnote\footnote%
\def\footnote{\protect\rmarkdownfootnote}

%%% Change title format to be more compact
\usepackage{titling}

% Create subtitle command for use in maketitle
\newcommand{\subtitle}[1]{
  \posttitle{
    \begin{center}\large#1\end{center}
    }
}

\setlength{\droptitle}{-2em}
  \title{Semana 2}
  \pretitle{\vspace{\droptitle}\centering\huge}
  \posttitle{\par}
  \author{Juan Carlos Girón}
  \preauthor{\centering\large\emph}
  \postauthor{\par}
  \predate{\centering\large\emph}
  \postdate{\par}
  \date{June 5, 2018}


\begin{document}
\maketitle

\subsection{Objetivos}\label{objetivos}

\begin{enumerate}
\def\labelenumi{\arabic{enumi}.}
\tightlist
\item
  Cargar Archivos de Texto en url a R.
\item
  Crear conexiones a url para cargar archivos Rdata.
\item
  Cargar archivos formato ``dat''.
\item
  Obtener registros con base al nombre de fila.
\end{enumerate}

\subsection{DAT file}\label{dat-file}

Time-Series Analyses of Beaver Body Temperatures Penny S. Reynolds.
Longitudinal data Model selection Regression diagnostics Sampling Time
series

\begin{Shaded}
\begin{Highlighting}[]
\NormalTok{df <-}\StringTok{ }\KeywordTok{read.table}\NormalTok{(}\StringTok{"http://www.stats.ox.ac.uk/pub/datasets/csb/ch11b.dat"}\NormalTok{,}\DataTypeTok{header =} \OtherTok{FALSE}\NormalTok{)}
\KeywordTok{head}\NormalTok{(df)}
\end{Highlighting}
\end{Shaded}

\begin{verbatim}
##   V1  V2   V3    V4 V5
## 1  1 307  930 36.58  0
## 2  2 307  940 36.73  0
## 3  3 307  950 36.93  0
## 4  4 307 1000 37.15  0
## 5  5 307 1010 37.23  0
## 6  6 307 1020 37.24  0
\end{verbatim}

\subsection{Txt file from http:}\label{txt-file-from-http}

Decathlon results

\begin{Shaded}
\begin{Highlighting}[]
\NormalTok{data<-}\KeywordTok{read.table}\NormalTok{(}\DataTypeTok{file=}\StringTok{"http://www.sthda.com/upload/decathlon.txt"}\NormalTok{) }
\KeywordTok{head}\NormalTok{(data)}
\end{Highlighting}
\end{Shaded}

\begin{verbatim}
##        V1    V2        V3       V4        V5    V6          V7     V8
## 1    name  100m Long.jump Shot.put High.jump  400m 110m.hurdle Discus
## 2  SEBRLE 11.04      7.58    14.83      2.07 49.81       14.69  43.75
## 3    CLAY 10.76       7.4    14.26      1.86 49.37       14.05  50.72
## 4  KARPOV 11.02       7.3    14.77      2.04 48.37       14.09  48.95
## 5 BERNARD 11.02      7.23    14.25      1.92 48.93       14.99  40.87
## 6  YURKOV 11.34      7.09    15.19       2.1 50.42       15.31  46.26
##           V9      V10   V11  V12    V13         V14
## 1 Pole.vault Javeline 1500m Rank Points Competition
## 2       5.02    63.19 291.7    1   8217           1
## 3       4.92    60.15 301.5    2   8122           1
## 4       4.92    50.31 300.2    3   8099           1
## 5       5.32    62.77 280.1    4   8067           1
## 6       4.72    63.44 276.4    5   8036           1
\end{verbatim}

\begin{Shaded}
\begin{Highlighting}[]
\KeywordTok{str}\NormalTok{(data)}
\end{Highlighting}
\end{Shaded}

\begin{verbatim}
## 'data.frame':    42 obs. of  14 variables:
##  $ V1 : Factor w/ 42 levels "Averyanov","Barras",..: 22 32 9 16 5 40 39 42 21 20 ...
##  $ V2 : Factor w/ 34 levels "10.44","10.5",..: 20 22 7 21 21 31 26 27 9 34 ...
##  $ V3 : Factor w/ 34 levels "6.61","6.68",..: 34 28 21 17 14 11 29 17 18 4 ...
##  $ V4 : Factor w/ 39 levels "12.68","13.07",..: 39 25 15 22 14 31 16 6 10 20 ...
##  $ V5 : Factor w/ 20 levels "1.85","1.86",..: 20 14 2 12 5 16 9 10 18 7 ...
##  $ V6 : Factor w/ 41 levels "400m","46.81",..: 1 29 21 5 12 33 9 8 30 31 ...
##  $ V7 : Factor w/ 37 levels "110m.hurdle",..: 1 19 4 5 28 34 9 7 14 24 ...
##  $ V8 : Factor w/ 40 levels "37.92","39.83",..: 40 19 38 35 8 30 10 28 20 31 ...
##  $ V9 : Factor w/ 19 levels "4.2","4.4","4.42",..: 19 14 12 12 17 8 12 3 3 12 ...
##  $ V10: Factor w/ 42 levels "50.31","50.62",..: 42 35 27 1 34 36 4 12 17 5 ...
##  $ V11: Factor w/ 39 levels "1500m","262.1",..: 1 35 38 37 28 20 23 9 31 2 ...
##  $ V12: Factor w/ 29 levels "1","10","11",..: 29 1 12 22 23 24 25 26 27 28 ...
##  $ V13: Factor w/ 40 levels "7313","7404",..: 40 30 29 27 24 23 22 18 17 10 ...
##  $ V14: Factor w/ 3 levels "1","2","Competition": 3 1 1 1 1 1 1 1 1 1 ...
\end{verbatim}

La primera fila del data set contiene el nombre de cada una de las
columnas. La estructura del data set muestra un error ya que cada
columna se encuentra en formato de factor,

\begin{Shaded}
\begin{Highlighting}[]
\CommentTok{#Agregar header TRUE para tener nombre de columnas.}

\NormalTok{data<-}\KeywordTok{read.table}\NormalTok{(}\DataTypeTok{file=}\StringTok{"http://www.sthda.com/upload/decathlon.txt"}\NormalTok{,}\DataTypeTok{header =}\NormalTok{ T) }
\KeywordTok{head}\NormalTok{(data)}
\end{Highlighting}
\end{Shaded}

\begin{verbatim}
##      name X100m Long.jump Shot.put High.jump X400m X110m.hurdle Discus
## 1  SEBRLE 11.04      7.58    14.83      2.07 49.81        14.69  43.75
## 2    CLAY 10.76      7.40    14.26      1.86 49.37        14.05  50.72
## 3  KARPOV 11.02      7.30    14.77      2.04 48.37        14.09  48.95
## 4 BERNARD 11.02      7.23    14.25      1.92 48.93        14.99  40.87
## 5  YURKOV 11.34      7.09    15.19      2.10 50.42        15.31  46.26
## 6 WARNERS 11.11      7.60    14.31      1.98 48.68        14.23  41.10
##   Pole.vault Javeline X1500m Rank Points Competition
## 1       5.02    63.19  291.7    1   8217           1
## 2       4.92    60.15  301.5    2   8122           1
## 3       4.92    50.31  300.2    3   8099           1
## 4       5.32    62.77  280.1    4   8067           1
## 5       4.72    63.44  276.4    5   8036           1
## 6       4.92    51.77  278.1    6   8030           1
\end{verbatim}

\begin{Shaded}
\begin{Highlighting}[]
\KeywordTok{str}\NormalTok{(data)}
\end{Highlighting}
\end{Shaded}

\begin{verbatim}
## 'data.frame':    41 obs. of  14 variables:
##  $ name        : Factor w/ 41 levels "Averyanov","Barras",..: 31 9 16 5 39 38 41 21 20 13 ...
##  $ X100m       : num  11 10.8 11 11 11.3 ...
##  $ Long.jump   : num  7.58 7.4 7.3 7.23 7.09 7.6 7.3 7.31 6.81 7.56 ...
##  $ Shot.put    : num  14.8 14.3 14.8 14.2 15.2 ...
##  $ High.jump   : num  2.07 1.86 2.04 1.92 2.1 1.98 2.01 2.13 1.95 1.86 ...
##  $ X400m       : num  49.8 49.4 48.4 48.9 50.4 ...
##  $ X110m.hurdle: num  14.7 14.1 14.1 15 15.3 ...
##  $ Discus      : num  43.8 50.7 49 40.9 46.3 ...
##  $ Pole.vault  : num  5.02 4.92 4.92 5.32 4.72 4.92 4.42 4.42 4.92 4.82 ...
##  $ Javeline    : num  63.2 60.1 50.3 62.8 63.4 ...
##  $ X1500m      : num  292 302 300 280 276 ...
##  $ Rank        : int  1 2 3 4 5 6 7 8 9 10 ...
##  $ Points      : int  8217 8122 8099 8067 8036 8030 8004 7995 7802 7733 ...
##  $ Competition : int  1 1 1 1 1 1 1 1 1 1 ...
\end{verbatim}

Esto soluciona el error de estructura. Podemos ver que la variable name
se exporta como factor y las restantes como número.

\subsubsection{Accesar a fila por nombre de
competidor.}\label{accesar-a-fila-por-nombre-de-competidor.}

\begin{Shaded}
\begin{Highlighting}[]
\NormalTok{data<-}\KeywordTok{read.table}\NormalTok{(}\DataTypeTok{file=}\StringTok{"http://www.sthda.com/upload/decathlon.txt"}\NormalTok{,}\DataTypeTok{header =}\NormalTok{ T, }\DataTypeTok{row.names =} \DecValTok{1}\NormalTok{,}\DataTypeTok{sep=}\StringTok{"}\CharTok{\textbackslash{}t}\StringTok{"}\NormalTok{) }

\NormalTok{data[}\StringTok{"CLAY"}\NormalTok{,]}
\end{Highlighting}
\end{Shaded}

\begin{verbatim}
##       X100m Long.jump Shot.put High.jump X400m X110m.hurdle Discus
## CLAY  10.76       7.4    14.26      1.86 49.37        14.05  50.72
##       Pole.vault Javeline X1500m Rank Points Competition
## CLAY        4.92    60.15  301.5    2   8122           1
\end{verbatim}

\subsubsection{Accesar a resultado
específico.}\label{accesar-a-resultado-especifico.}

\begin{Shaded}
\begin{Highlighting}[]
\NormalTok{data[}\StringTok{"CLAY"}\NormalTok{,}\StringTok{"Shot.put"}\NormalTok{]}
\end{Highlighting}
\end{Shaded}

\begin{verbatim}
## [1] 14.26
\end{verbatim}

\subsection{Ejercicio:}\label{ejercicio}

El data set presenta nombres duplicados en mayúscula y minúscula. Crear
un formato de nombres en el cual cada registro se encuentre en
mayúsculas y no se presenten duplicados.

Nota: No pueden existir nombres de fila duplicados.


\end{document}
